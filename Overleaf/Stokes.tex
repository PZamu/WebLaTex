\documentclass[11pt]{article}

\marginparwidth 0.5in 
\oddsidemargin 0.25in 
\evensidemargin 0.25in 
\marginparsep 0.25in
\topmargin -0.1in 
\textwidth 6in \textheight 8.1in

\usepackage{upgreek}
\usepackage[utf8]{inputenc}
\usepackage{amsfonts}
\usepackage{amsmath}
\usepackage{mathtools}
\usepackage{mathpazo}
\usepackage{multicol}
%\usepackage{units}
\usepackage{amssymb}
\usepackage{physics}
\usepackage{amsthm}
\usepackage{setspace}
\usepackage{fancybox}
\usepackage[spanish]{babel}
\usepackage[T1]{fontenc}
%\usepackage[pdftex]{graphicx}
\usepackage{cancel}
\usepackage{fancyhdr}
\usepackage{eurosym}
\usepackage[dvips]{graphicx}
%\usepackage{arcs}
%\DeclareGraphicsExtensions{.png,.pdf,.jpg}
\usepackage[colorlinks=true,linkcolor=blue]{hyperref}
\usepackage{pst-3dplot,pstricks-add}
\usepackage{auto-pst-pdf}
\usepackage[colorlinks=true]{hyperref}
\hypersetup{
    colorlinks,%
    citecolor=black,%
    filecolor=black,%
    linkcolor=black,%
    urlcolor=black,
%   hyperindex,
    breaklinks=true
}
\usepackage{polynom}
\usepackage{verbatim}
\parskip=1.5mm
\usepackage{wrapfig}
\usepackage{color,colortbl}
\usepackage{yhmath}  %\widetriangle{ABC}
%\usepackage{sagetex}
\usepackage{polynom}
\usepackage{tikz}
\usetikzlibrary{calc,decorations.markings}
\usepackage{pgfplots}
\usepackage[notquote]{hanging} %hangparas para sangría francesa. La opción [notquote] se puso para que no hubiera problemas a la hora de escribir derivadas en modo matemático como \(f'\) en lugar de tener que utilizar \(f^{\prime}\)
\usepackage{qrcode}
\usepackage{mdframed}
\usepackage{enumitem}
\usepackage{mathtools}
\usepackage{titling}
\preauthor{}
\postauthor{}

%%%%%%%Para alargar (sin engordar) símbolos de integrales
\usepackage{scalerel}
\def\stretchint#1{\vcenter{\hbox{\stretchto[440]{\displaystyle\int}{#1}}}}
%%%%%%%%%%%%%%%%%%%%%%%%%%%%%%%%%%%%%%%%%%%%%%%%%%%%%%%%%%%
%%%%%%%%%%%%%%%% \stretchint{7ex} \frac{x}{2}\, dx aumentando {7ex} estiramos hacia arriba y abajo la integral

\newtheorem{thm}{Teorema}
\newtheorem{prop}{Proposición}
\newcommand{\mcd}{\textrm{mcd}}
\newcommand{\R}{\mathbb{R}}
\newcommand{\C}{\mathbb{C}}
\newcommand{\N}{\mathbb{N}}
\newcommand{\Z}{\mathbb{Z}}
\newcommand{\la}{\lambda}
\newcommand{\al}{\alpha}
\newcommand{\om}{\omega}
\newcommand{\g}{\gamma}
\newcommand{\G}{\Gamma}
\newcommand{\rotaF}{\nabla \times \mathbf{F}}
\newcommand{\res}{\textrm{Res}}
\newcommand{\ii}[2]{\int\limits_{#1}^{#2}}
% draw a frame around given text
\newcommand{\framedtext}[1]{%
\par%
\noindent\fbox{%
    \parbox{\dimexpr\linewidth-2\fboxsep-2\fboxrule}{#1}%
}%
}
%%%%%% Para no indentar notas al pie %%%%%%%
\makeatletter

\renewcommand\@makefntext[1]{%
\setlength\parindent{1em}%
\noindent
\makebox[0.7em][r]{\textsuperscript{\@thefnmark}~}{#1}}

\makeatother
%%%%%%%%%%%%%%%%%%%%%%%%%%%%%%%%%%%%%%%%%%%%

%\definecolor{verdeodi}{RGB}{53,113,105}
\definecolor{verdeodi}{RGB}{0,122,255}
\usepackage[most]{tcolorbox}
\newtcolorbox{fancybox}[1][]{
	enhanced,
	boxrule=1pt,arc=8pt,boxsep=0pt,
	left=0.8em,right=0.8em,top=1ex,bottom=1ex,colback=verdeodi!20,colframe=verdeodi,#1
}

\newtcolorbox{fancybox2}[1][]{
	enhanced,
	boxrule=1pt,arc=8pt,boxsep=0pt,
	left=0.8em,right=0.8em,top=1ex,bottom=1ex,colback=verdeodi!10,colframe=white,#1
}

\newtcolorbox{Box2}[2][]{
                lower separated=false,
                colback=white,
colframe=verdeodi,fonttitle=\bfseries,
colbacktitle=verdeodi,
coltitle=white,
enhanced,
attach boxed title to top left={yshift=-0.1in,xshift=0.15in},
                 boxed title style={boxrule=0pt,colframe=white,},
title=#2,#1}



\makeatletter%%%%%%%%para referenciar ecuaciones en línea
\newcommand*{\inlineequation}[2][]{%
  \begingroup
    % Put \refstepcounter at the beginning, because
    % package `hyperref' sets the anchor here.
    \refstepcounter{equation}%
    \ifx\\#1\\%
    \else
      \label{#1}%
    \fi
    % prevent line breaks inside equation
    \relpenalty=10000 %
    \binoppenalty=10000 %
    \ensuremath{%
      % \displaystyle % larger fractions, ...
      #2%
    }%
    \hfill~\@eqnnum %%%%si quieres que aparezca al lado de la ecuación quita \hfill
  \endgroup
}
\makeatother

%%%%% Genera los pies de Figuras y Tablas  
\usepackage[format=plain, labelformat=simple, labelsep=period, font={color=verdeodi,sf},labelfont=bf,centerlast]{caption}
\renewcommand\thefigure{\arabic{figure}} % Genera numeración X
\renewcommand\thetable{\arabic{table}} % Genera numeración X
%%%%%%%%%%%%%%%%%%%%%%%%%%%%%%%%%%%%%%%%%%%%%%%%





\title{\vspace{-20mm} Reto cálculo vectorial }
%\author{\href{mailto:pvitoria@gmail.com}{Pablo Vitoria García}}
\date{\today}

\newenvironment{Figura}
{\par\medskip\noindent\minipage{\linewidth}}{\endminipage\par\medskip}

\newcommand\miazulbox[2][]{\tikz[overlay]\node[fill=verdeodi!10,inner xsep=2pt, inner ysep=4pt, anchor=text, rectangle, rounded corners=3pt,#1] {\ #2\ \mbox{}};\phantom{#2}}
\usepackage{empheq}

\usepackage{xparse}

\newlength {\longexp} % longitud del exponente
\newcommand{\diff}{\mathop{}\!d}
\newcommand{\newint}[2]{\int_{\lower0.7ex\hbox{$\scriptstyle#1$}}^{\raise0.7ex\hbox{$\scriptstyle#2$}}}
\NewDocumentCommand{\otraint}{e{^_}}
{%
  \mathop{}\!%
  \int%
  \IfValueT{#2}{_{\kern-0.2em\lower0.8ex\hbox{$\scriptstyle#2$}}}%
  \IfValueT{#1}{^{\kern-0.0em\raise0.5ex\hbox{$\scriptstyle#1$}}\settowidth{\longexp}{$#1$}\hspace{-0.75\longexp}}%
  \!\mathop{}%
}


\begin{document}
\renewcommand{\tablename}{Tabla}

\maketitle

\thispagestyle{empty}


\begin{fancybox}%%%%%%%%%%%%%%%%%%%%%%%%%%%%%%%%%%%%%%%%%%%%%%%
    %\noindent\textbf{Problema:} 

    Una partícula recorre una trayectoria dada en coordenadas cartesianas por \((\cos t, \sen t, a-\cos^2 t - \sen t)\) con \(a=2\), y \(t\) desde \(0\) hasta \(2\pi\), en una región del espacio donde existe un campo de fuerzas \(\mathbf{F}(x,y,z) = (z^2-y^2, -2xy^2, e^{\sqrt{z}}\cos z)\). Calcular el trabajo realizado por la fuerza sobre la partícula a lo largo de la trayectoria.

    Considerando ahora el parámetro \(a\) variable, ¿podrías calcular el trabajo realizado como función de \(a\)?

\end{fancybox}%%%%%%%%%%%%%%%%%%%%%%%%%%%%%%%%%%%%%%%%%%%%%%%% 

%\noindent%
{\sf Propuesto por Pablo Vitoria García. Resuelto por Pablo Vitoria García.}

%\vspace{3mm}

El trabajo realizado por el campo de fuerzas (campo vectorial) \(\mathbf{F}\) a lo largo de la trayectoria dada, parametrizada por \(\g(t)=(\cos t, \sen t, a-\cos^2 t - \sen t)\) viene dado por la integral de línea del campo a lo largo de \(\g\), \(\int_{\g} \mathbf{F} \).

Dada la complejidad del campo vectorial y de la trayectoria; y teniendo en cuenta que el rotacional del campo, \(\rotaF\), es mucho más simple:
\[ \rotaF = \left( \frac{\partial F_z}{y}-\frac{\partial F_y}{z}, \frac{\partial F_x}{z}-\frac{\partial F_z}{x}, \frac{\partial F_y}{x}-\frac{\partial F_x}{y} \right) = (0, 2z,-2y^2+2y)\]
se aplicará el \textsc{Teorema de Stokes}:


\framedtext{
    Sea \(S\) una superficie abierta, orientada y de clase \(C^1\) a trozos, cuya frontera, \(\partial S\), es una curva de clase \(C^1\) a trozos, simple y cerrada, con orientación positiva. Si \(\mathbf{F}\) es un campo vectorial de clase \(C^1\) en un abierto del espacio que contiene a \(S\), entonces se tiene la siguiente igualdad:
    \[ \int_{\partial S} \mathbf{F} = \iint_{S} \rotaF \]
}


Comprobemos si se cumplan las condiciones necesarias para applicar el teorema.
\begin{itemize}
    \item El campo \( \mathbf{F}\), debido a la presencia de la raíz cuadrada en su componente \(z\), solo está definido  en el dominio abierto de \(\R^3\) con \(z>0\), donde es de clase \(C^1\).
    \item Por tanto la trayectoria tiene que tener todos sus puntos en ese dominio para que tenga sentido calcular el trabajo.
          Dado que la curva es acotada, el mínimo de la coordenada \(z(t)\) se obtiene para algún valor del parámetro \(t\) tal que \(z'(t)= (2 \sen t -1)\cos{t}=0\). Es decir, para \(t=\pi/6, \pi/2, 5\pi/6 \ \textrm{o}\  3\pi/2\). Basta evaluar en ellos \(z(t)\) para determinar que el valor mínimo es \(z(\pi/6)=z(\pi/6)=a-5/4\). Así que \emph{solo tiene sentido calcular el trabajo realizado si \(a>5/4\)}.
    \item La parametrización \( \g(t)\) de la trayectoria es de clase \(C^1\). Y la trayectoria es:
          \begin{itemize}
              \item \emph{cerrada} (ya que \(\g(0) = \g(2 \pi)=(1,0,1)\));
              \item y \emph{simple}, es decir, inyectiva en el intervalo \((0,2\pi)\), ya que con el cambio de variable \((x,y)=(\cos{t},\sen t)\) su expresión en cartesianas es \((x,y,a-x^2-y)\), con \((x,y)\) sobre la circunferencia \(x^2+y^2=1\). Es decir, se trata de una curva definida sobre la circunferencia unidad.
          \end{itemize}
    \item Por otro lado, la superficie \(S\) definida como el trozo de cilindro parabólico \footnote{Tenía que haber parametrizado la superficie como \((x,y,z)=(r \cos t,r \sen t,a- \cos^2t - \sen t)\) con \(0 < r < 1\) y \( 0 < t < 2 \pi\). Mucho más cómodo ya que se evita el uso posterior de coordenadas polares. Pero no me da tiempo a cambiarlo ahora...}: \(z=a-x^2-y\) con \((x,y) \in R\), siendo \(R\) el interior de la circunferencia \(x^2+y^2=1\) es de clase \(C^1\). Y su frontera, \(\partial S\), es la trayectoria \(\g(t)\).
\end{itemize}

Como conclusión, se puede aplicar el teorema de Stokes para calcular el trabajo.

Dado que la trayectoria se recorre en sentido positivo (antihorario), la integral de superficie del rotacional de \(\mathbf{F}\) hay que realizarla a través de la cara superior de \(S\) (regla de la mano derecha, dejar la superficie a la izquierda,...). Por tanto se puede tomar como vector normal, \(\mathbf{N}\), a \(S\) al vector:
\[\mathbf{N}=(-z'_x,-z'_y,1)= (2x,1,1)\]

Y se tiene que:
\begin{align*}
    \int_{\g} \mathbf{F} & = \iint_{S} \rotaF = \iint_{R} (\rotaF)(S) \cdot \mathbf{N} \ dxdy                         \\
                         & = \iint_{R} (0,2(a-x^2-y),-2y^2-2y) \cdot (2x,1,1)\ dxdy = \iint_{R} (2a-2x^2-2y^2) \ dxdy
\end{align*}
Introduciendo coordenadas polares \((x,y)=(\rho \cos \theta, \rho \sen \theta)\), con jacobiano \(\rho\), se tiene que la región \(R\) viene definida por \(0<\rho<1,\quad 0<\theta<2 \pi\):
\begin{align*}
    \int_{\g} \mathbf{F} = \int_0^{2 \pi} \int_0^1 (a- \rho^2) 2\rho\ d\rho d\theta = -\pi \left[(a-\rho^2)^2\right]_0^1=(2a-1)\pi
\end{align*}


Como conclusión, el trabajo realizado por el campo \(\mathbf{F}\) a lo largo de esa trayectoria es:
\framedtext{
    \[(2a-1) \pi \quad \textrm{para}\quad a > \frac{5}{4}\]
}
\end{document}



\begin{align} \label{finv}
    f(0) & \sim \mathcal{F}^{-1}[F](t)=\frac{1}{2\pi}\ii{-\infty}{\infty}F(x)\ \mathrm{e}^{i0x}\ dx= \frac{1}{2\pi}\ii{-\infty}{\infty}F(x) (\cos(tx)+i\sen(tx))\ dx \notag                                             \\
         & =\frac{1}{2\pi}\ii{-\infty}{\infty}F(x) \cos(tx)\ dx + \underbrace{\frac{i}{2\pi}\ii{-\infty}{\infty}F(x) \sen(tx)\ dx}_{0, F(x) \sen(tx) \textrm{ es impar}} = \frac{1}{\pi}\ii{0}{\infty}F(x) \cos(tx)\ dx
\end{align}

donde en el último paso se ha usado que \(F(x) \cos(tx)\) es par con respecto a \(x\).































